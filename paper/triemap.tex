% Links
% https://stackoverflow.com/questions/16084788/generic-trie-haskell-implementation
% Hinze paper: http://citeseerx.ist.psu.edu/viewdoc/summary?doi=10.1.1.8.4069

%% For double-blind review submission, w/o CCS and ACM Reference (max submission space)
% \documentclass[acmsmall,review]{acmart}\settopmatter{printfolios=true,printccs=false,printacmref=false}
%% For double-blind review submission, w/ CCS and ACM Reference
%\documentclass[acmsmall,review,anonymous]{acmart}\settopmatter{printfolios=true}
%% For single-blind review submission, w/o CCS and ACM Reference (max submission space)
% \documentclass[acmsmall,review]{acmart}\settopmatter{printfolios=true,printccs=false,printacmref=false}
%% For single-blind review submission, w/ CCS and ACM Reference
%\documentclass[acmsmall,review]{acmart}\settopmatter{printfolios=true}
%% For final camera-ready submission, w/ required CCS and ACM Reference
\documentclass[acmsmall]{acmart}

%% Journal information
%% Supplied to authors by publisher for camera-ready submission;
%% use defaults for review submission.
% \acmJournal{PACMPL}
% \acmVolume{1}
% \acmNumber{ICFP} % CONF = POPL or ICFP or OOPSLA
% \acmArticle{1}
% \acmYear{2020}
% \acmMonth{8}
% \acmDOI{} % \acmDOI{10.1145/nnnnnnn.nnnnnnn}
% \startPage{1}

%% Copyright information
%% Supplied to authors (based on authors' rights management selection;
%% see authors.acm.org) by publisher for camera-ready submission;
%% use 'none' for review submission.
% \setcopyright{none}
%\setcopyright{acmcopyright}
%\setcopyright{acmlicensed}
%\setcopyright{rightsretained}
%\copyrightyear{2018}           %% If different from \acmYear

%%% The following is specific to ICFP '20 and the paper
%%% 'Kinds Are Calling Conventions'
%%% by Paul Downen, Zena M. Ariola, Simon Peyton Jones, and Richard A. Eisenberg.
%%%
\setcopyright{rightsretained}
\acmPrice{}
\acmDOI{10.1145/3408986}
\acmYear{2020}
\copyrightyear{2020}
\acmSubmissionID{icfp20main-p93-p}
\acmJournal{PACMPL}
\acmVolume{4}
\acmNumber{ICFP}
\acmArticle{104}
\acmMonth{8}

%% Bibliography style
\bibliographystyle{ACM-Reference-Format}
%% Citation style
%% Note: author/year citations are required for papers published as an
%% issue of PACMPL.
\citestyle{acmauthoryear}   %% For author/year citations


%%%%%%%%%%%%%%%%%%%%%%%%%%%%%%%%%%%%%%%%%%%%%%%%%%%%%%%%%%%%%%%%%%%%%%
%% Note: Authors migrating a paper from PACMPL format to traditional
%% SIGPLAN proceedings format must update the '\documentclass' and
%% topmatter commands above; see 'acmart-sigplanproc-template.tex'.
%%%%%%%%%%%%%%%%%%%%%%%%%%%%%%%%%%%%%%%%%%%%%%%%%%%%%%%%%%%%%%%%%%%%%%


%% Some recommended packages.
\usepackage{booktabs}   %% For formal tables:
                        %% http://ctan.org/pkg/booktabs
\usepackage{subcaption} %% For complex figures with subfigures/subcaptions
                        %% http://ctan.org/pkg/subcaption
\usepackage[utf8]{inputenc}
\usepackage{microtype}
\usepackage{amsmath}
\usepackage{amsthm}
% \usepackage{amssymb}
% \usepackage{stmaryrd}
\usepackage{framed}
\usepackage{proof}
\usepackage{braket}
\usepackage{fancyvrb}
\usepackage{listings}
\usepackage[inline,shortlabels]{enumitem}
\usepackage[capitalize]{cleveref}
\usepackage{xcolor}
\usepackage{pgffor}
\usepackage{ragged2e}

% \RequirePackage{xargs}

\VerbatimFootnotes

\lstset{language=Haskell}

\let\restriction\relax

\theoremstyle{theorem}
\newtheorem{proposition}{Proposition}
\newtheorem{lemma}{Lemma}
\newtheorem{theorem}{Theorem}
\newtheorem{corollary}{Corollary}
\newtheorem{property}{Property}
\theoremstyle{definition}
\newtheorem{definition}{Definition}
\newtheorem{restriction}{Restriction}
\newtheorem{intuition}{Intuition}
\theoremstyle{remark}
\newtheorem{remark}{Remark}
\newtheorem{notation}{Notation}

%% Style guide forbids boxes around figures
% \newenvironment{figurebox}{\begin{figure}\begin{framed}}{\end{framed}\end{figure}}
% \newenvironment{figurebox*}{\begin{figure*}\begin{framed}}{\end{framed}\end{figure*}}
\newenvironment{figurebox}{\begin{figure}}{\end{figure}}
\newenvironment{figurebox*}{\begin{figure*}}{\end{figure*}}

\crefname{figure}{Fig.}{Figs.}
\Crefname{figure}{Fig.}{Figs.}
\crefname{restriction}{Restriction}{Restrictions}

% Some colors:
\definecolor{dkcyan}{rgb}{0.1, 0.3, 0.3}
\definecolor{dkgreen}{rgb}{0,0.3,0}
\definecolor{olive}{rgb}{0.5, 0.5, 0.0}
\definecolor{dkblue}{rgb}{0,0.1,0.5}

\definecolor{col:ln}{rgb}  {0.1, 0.1, 0.7}
\definecolor{col:str}{rgb} {0.8, 0.0, 0.0}
\definecolor{col:db}{rgb}  {0.9, 0.5, 0.0}
\definecolor{col:ours}{rgb}{0.0, 0.7, 0.0}

\definecolor{lightgreen}{RGB}{170, 255, 220}
\definecolor{darkbrown}{RGB}{121,37,0}

% Customized syntax highlighting for Haskell code snippets:
\colorlet{listing-comment}{gray}
\colorlet{operator-color}{darkbrown}

\lstdefinestyle{default}{
    basicstyle=\ttfamily\fontsize{8.7}{9.5}\selectfont,
    columns=fullflexible,
    commentstyle=\sffamily\color{black!50!white},
    escapechar=\#,
    framexleftmargin=1em,
    framexrightmargin=1ex,
    keepspaces=true,
    keywordstyle=\color{dkblue},
    mathescape,
    numbers=none,
    numberblanklines=false,
    numbersep=1.25em,
    numberstyle=\relscale{0.8}\color{gray}\ttfamily,
    showstringspaces=true,
    stepnumber=1,
    xleftmargin=1em
}

\lstdefinelanguage{custom-haskell}{
    language=Haskell,
    deletekeywords={lookup, delete, map, mapMaybe, Ord, Maybe, String, Just, Nothing, Int, Bool},
    keywordstyle=[2]\color{dkgreen},
    morekeywords=[2]{String, Map, Ord, Maybe, Int, Bool},
    morekeywords=[2]{Name, Expression, ESummary, PosTree, Structure, HashCode, VarMap},
    keywordstyle=[3]\color{dkcyan},
    literate=%
        {=}{{{\color{operator-color}=}}}1
        {|}{{{\color{operator-color}|}}}1
        {\\}{{{\color{operator-color}\textbackslash$\,\!$}}}1
        {.}{{{\color{operator-color}.}}}1
        {=>}{{{\color{operator-color}=>}}}1
        {->}{{{\color{operator-color}->}}}1
        {<-}{{{\color{operator-color}<-}}}1
        {::}{{{\color{operator-color}::}}}1
}

\lstset{style=default}
% Environment for code snippets
\lstnewenvironment{code}[1][]
  {\small\lstset{language=custom-haskell,#1}}
  {}

% Environment for example expressions
\lstnewenvironment{expression}[1][]
  {\small\lstset{#1}}
  {}

\begin{document}

\newcommand{\simon}[1]{{\bf SLPJ}: #1 {\bf End SLPJ}}

\newcommand{\bv}[1]{\#_{#1}}    -- Lambda-bound variable occurrence
\newcommand{\pv}[1]{\$_{#1}}    -- Pattern variable binder
\newcommand{\pvo}[1]{\%_{#1}}   -- Pattern variable occurrence


%% Title information
\title%[Short Title]
{Triemaps that match}         %% [Short Title] is optional;
                                        %% when present, will be used in
                                        %% header instead of Full Title.
% \titlenote{with title note}             %% \titlenote is optional;
%                                         %% can be repeated if necessary;
%                                         %% contents suppressed with 'anonymous'
% \subtitle{Subtitle}                     %% \subtitle is optional
% \subtitlenote{with subtitle note}       %% \subtitlenote is optional;
%                                         %% can be repeated if necessary;
%                                         %% contents suppressed with 'anonymous'


%% Author information
%% Contents and number of authors suppressed with 'anonymous'.
%% Each author should be introduced by \author, followed by
%% \authornote (optional), \orcid (optional), \affiliation, and
%% \email.
%% An author may have multiple affiliations and/or emails; repeat the
%% appropriate command.
%% Many elements are not rendered, but should be provided for metadata
%% extraction tools.

%% Author with single affiliation.
\author{Simon Peyton Jones}
\affiliation{
  \institution{Microsoft Research}
  % \streetaddress{21 Station Rd.}
  \city{Cambridge}
  % \postcode{CB1 2FB}
  \country{UK}
}
\email{simonpj@microsoft.com}

\author{Richard A.~Eisenberg}
\affiliation{
  \institution{Tweag I/O}
  \city{Cambridge}
  \country{UK}
}
\email{rae@richarde.dev}

\author{Josef Sveningsson}

%% Abstract
%% Note: \begin{abstract}...\end{abstract} environment must come
%% before \maketitle command
\begin{abstract}
TrieMaps are great.
\end{abstract}

%% \maketitle
%% Note: \maketitle command must come after title commands, author
%% commands, abstract environment, Computing Classification System
%% environment and commands, and keywords command.
\maketitle


\section{Introduction} \label{sec:intro}

The designs of many programming languages include a feature where some concrete use of a construct is matched against a set of possible interpretations, where the possible interpretations might be defined in terms of variables to be instantiated at concrete usages. For example:
\begin{itemize}
\item Haskell's class instances work this way: the user defines instances (which may contain variables), and then concrete usage sites of class methods require finding an instance that applies.
\item Agda, Coq, and Idris all have implicit-argument features directly inspired by Haskell's class mechanism.
\item An extension to Haskell allows overloaded instances, where we select the most specific instance (that is, one that is a specialization of any other possible matching instance).
\item C++ and Java both support function overloading, where a usage site of a function must be matched against a choice of implementation. C++'s templates and Java's generics allow for variables to be used in the implementations. There are sometimes multiple implementations that match; both languages choose the most specific match (for an appropriate definition of "most specific").
\item C++ separately allows template specialization, where a templated definition can have concrete specializations. Once again, selection of these specializations depends on a "most-specific" relation appropriate to the case.
\item Scala? C\#?
\end{itemize}
Beyond features specified in a language's design, optimizations may require such a structure. For example, GHC's rewrite rules~\cite{rewrite-rules} requires a similar lookup to find a mapping from expressions to rules that may apply.

Our concern here is the efficient implementation of this matching
operation. That is, we wish to define a structure mapping keys to
arbitrary values. The keys are chosen from a type described by a
context-free grammar and thus comprise small trees. We assume a total
ordering on such trees. The challenge lies in the fact that we want
our structure to support wildcard variables, which represent any tree
at all. We will write such variables with Greek letters. Accordingly,
we can view a mapping \lstinline{alpha |-> v} as an infinite mapping, connecting
all possible trees to v, or we can have \lstinline{Maybe alpha |-> v2} map all
trees whose root is Maybe to the value v2. Combining these to `alpha
|-> v, Maybe alpha |-> v2` would map all trees to v, except those
trees that have Maybe at the root, which map to v2. Accordingly,
looking up in our structure find the most specific
match.\footnote{This aspect of our design is unforced. We could also
return all possible matches, instead of selecting the most
specific. See \cref{sec:most-specific}.}

Our contributions are as follows:

\begin{itemize}
\item We describe the language-agnostic Variable Trie-Map (VTM) data structure, with the semantics as described above (and made more precise in \cref{sec:vtm}). The keys in a VTM support local bound variables via on-the-fly conversion to de Bruijn levels, necessary to support polymorphic types or $\lambda$-expressions. Looking up in a VTM is linear in the size of the key, regardless of the size of the VTM itself. The operations on a VTM are proved to uphold the sensible properties described in \cref{sec:vtm-properties}.
\item Some languages require not only the matching behavior above, but also a \emph{unification} operation, where we find not only keys that match, but keys that would match if variables in the looked-up tree were further instantiated. \Cref{sec:unification} describes how we extend VTMs to support unification as well as matching.
\item While VTMs are time-efficient, they can be space-inefficient. \Cref{sec:path-compression} describes an easy optimization which drastically reduces the memory footprint of VTMs whose keys share little structure in common. This optimization shows a 30% geometric mean savings in the size of the structure GHC uses to look up rewrite rules.
\item Our work was motivated by quadratic behavior in GHC, observed when checking for instance consistency among imports. This became a problem in practice in Haskell's use at Facebook. We report on our implementation within GHC, showing that it achieves a 98% speedup against the previous (admittedly naive) implementation of instance tables.
\end{itemize}

\section{The problem we address}
\begin{figurebox}
\begin{code}
type XT v = Maybe v -> Maybe v

Map.empty  :: Map k v
Map.insert :: Ord k => k -> v -> Map k v
Map.lookup :: Ord k => k -> Map k v -> Maybe v
Map.alter  :: Ord k => k -> XT v -> Map k v -> Maybe v
Map.union  :: Ord k => Map k a -> Map k a -> Map k a
Map.size   :: Map k v -> Int
Map.foldr  :: (v -> r -> r) -> r -> Map k v -> r


infixr 1 (>=>)  -- Kleisli composition
(>=>) :: Monad m => (a -> m b) -> (b -> m c) -> a -> m c

infixr 1 (>.>)   -- Forward composition
(>.>)  :: (a -> b) -> (b -> c) -> a -> c

infixr 0 (|>)   -- Reverse function application
(|>)  :: a -> (a -> b) -> b
\end{code}
\caption{API for library functions}
\label{fig:containers} \label{fig:library}
\end{figurebox}

Our general task is as follows: \emph{implement a finite mapping from keys to values,
in which the key type is some kind of tree}.
For example, an \lstinline{Expr} data type might be defined like this:
\begin{code}
data Expr = App Expr Expr | Lam  Var Expr | Var Var
\end{code}
Here \lstinline{Var} is the type of variables; which can be compared for
equality and used as the key of a finite map.  Its definition is nor important
for this paper, but for the sake of concretemess
you could imagine it being defined simply as a string:
\begin{code}
  type Var = String
\end{code}
The data type \lstinline{Expr} is capable of representing expressions like $add 2 3$
$\lambda x. add x 3$.  We will use this data type throughout the paper, because it
has all the features that occur in real expression data types: free variables like $add$, represented by a \lstinline{Var} node;
lambdas which can bind variables (\lstinline{Lam}), and occurrences of those bound variables (\lstinline{Var}); 
and nodes with multiple children (\lstinline{App}).  A real-world compiler like GHC would have
many more constructors in the data type.

A finite map keyed by such expressions is extremely useful.
GHC uses such a map during its common sub-expression
elimination pass, where the map associates an
expression with the identifier bound to that expression; if the same
expression occurs again, we can look it up in the map, and replace the
expression with the variable.
GHC also does many lookups based on \emph{types} rather than
\emph{expressions}.  For example, when implementing type-class
instance lookup, or doing type-family reduction, GHC needs a map whose
key is a type.

\subsection{The API of of a finite map}

What API might such a map have? We follow the design pattern of
the \lstinline{containers} library, whose key functions are given in \Cref{fig:containers},
and seek these basic operations:
\begin{code}
  type XT v = Maybe v -> Maybe v

  empty  :: ExprMap v
  lookup :: Expr -> ExprMap v -> Maybe v
  alter  :: Expr -> XT v -> ExprMap v -> ExprMap v
\end{code}
The functions \lstinline{empty} and \lstinline{lookup} should be
self-explanatory.  The function \lstinline{alterTM} is a standard
generalisation of \lstinline{insert}, in which the element to be
inserted is generalised to a function \lstinline{XT v}, an
abbreviation for \lstinline{Maybe v -> Maybe v}.  This function
transforms the existing value associated with key, if any (hence the
\lstinline{Maybe}), to a new value, with \lstinline{Nothing}
indicating deletion.  Given \lstinline{alter} we can easily define \lstinline{insert} and \lstinline{delete}:
\begin{code}
  insert :: Expr -> v -> ExprMap v -> ExprMap v
  insert e v = alter e (\_ -> Just v)

  delete :: Expr -> ExprMap v -> ExprMap v
  delete e = alter e (\_ -> Nothing)
\end{code}
You might wonder if, for the purposes of this paper we could just define \lstinline{insert},
leaving \lstinline{alter} for the Appendix, but as we will see in \Cref{sec:alter}, our
approach using tries fundamentally requires the generality of \lstinline{alter}.

We would also like to support other standard operations on finite maps, including
\begin{itemize}
\item An efficient union operation to combine two finite maps into one.
\item A map operation to apply a function to the range of the finite map.
\item A fold operation to combine together the elements of the range.
\end{itemize}

\subsection{Alpha-renaming}

Remember that the type \lstinline{Expr} is the \emph{key} of our \lstinline{ExprMap} type.
So we would expect insertion and lookup to be insensitive to
$\alpha$-renaming. For example suppose we insert a value with key
$(\lambda x.\, add~ x~ 3)$, and then look up the
key $(\lambda y.\, add~ y~ 3)$: we should
certainly expect to find the value we inserted, even though the name
of the $\lambda$-bound variable has changed.


\subsection{Matching} \label{sec:matching-intro}

In GHC we want more: we want to a lookup that does \emph{matching}.  GHC supports
so-called \emph{rewrite rules} \cite{rewrite-rule-paper}, which the user can specify like this:
\begin{code}
{-# RULES "map/map" forall f g xs. map f (map g xs) = map (f . g) xs #-}
\end{code}
This rule asks the compiler to rewrite any call that matches the shape
of the left-hand side (LHS) of the rule into the right-hand side
(RHS).  We often use the term \emph{pattern} to describe the LHS.
The pattern is explicitly quantified over the \emph{pattern variables}
(here \lstinline{f}, \lstinline{g}, and \lstinline{xs}) that
that can be bound during the matching process.  In other words, \emph{we seek a substitution
for the pattern variables that makes the pattern equal to the target expression}.

Now imagine that we have thousands of such rules.  Given a target
expression, we want to look up in the rule database, to see if any
rule matches.  One approach would be to look at the rules one at a
time, checking for a match, but that would be slow if there are many rules.
How could we do this more efficiently?

Of course, the pattern might itself have bound variables, and we would
like to be insensitive to alpha-conversion for those. For example:
\begin{code}
{-# RULES "map/id"  map (\\x -> x) = \\x -> x
\end{code}
We want to find a successful match if we see a call \lstinline{(map (\\y -> y))},
even though the bound variable has a different name.

Lookup-modulo-matching is a valuable ability.  For example, GHC's lookup for
type-class instances, and for type-family instances, can again have thousands
of candidate, and depends crucially on matching.

It is also an ability that is hard to reconcile with many standard approaches to
implementing finite maps.   For example, representing the finite map as a binary tree,
and performing comparisons at the nodes to determine which sub-tree holds the key,
seems an approach that is hard or impossible to extend to support matching.

\subsection{Non-solutions} \label{sec:ord}

At first sight the job can be done easily: define a total order on
\lstinline{Expr}, make it an instance of the class \lstinline{Ord},
and use a standard finite map library such as \lstinline{Data.Map}.
And indeed that works, but it is terribly slow.  A finite map is
implemented as a binary search tree; at every node of this tree, we compare the key (a \lstinline{Expr}, remember) with
a key stored at the node; if it is smaller, go left; if larger, go right. So each lookup
performs a (logarithmic) number of potentially-full-depth comparisons of two expressions.

Another possibility might be to hash the \lstinline{Expr} and use the
hash-code as the lookup key.  That would make lookup much faster, but
it requires at least two full traversals of the key for every lookup:
one to compute its hash code for every lookup, and a full equality
comparison on a ``hit'' because hash-codes can collide.  Moreover
hashing algorithms that respect $\alpha$-conversion are not trivial
\cite{alpha-hashing}.

\section{Tries} \label{sec:ExprS}

A rather standard approach to a finite map in which the key has internal structure
is to use a \emph{trie} \cite{trie}.  Let us consider a simplified
form of expression:
\begin{code}
data ExprS = VarS Var | AppS ExprS ExprS
\end{code}
We have lambdas for now, so that all \lstinline{Var} nodes represent free variables, which are constants.
We will return to lambdas in \Cref{sec:binders}.

\subsection{The basic idea} \label{sec:basic} \label{sec:alter}

Here is a trie-based implemenation for \lstinline{ExprS}:
\begin{code}
data ExprSMap v = ESM { esm_var :: Map Var v
                      , esm_app :: ExprSMap (ExprSMap v) }
\end{code}
Here \lstinline{Map Var v} is any standard, existing finite map keyed by \lstinline{Var}.
There are many such finite maps available for an ordered type like \lstinline{Var};
the \lstinline{containers} package will do as well as any.

One way to understand this slightly odd data type is to study its lookup function:
\begin{code}
lookupExprS :: ExprS -> ExprSMap v -> Maybe v
lookupExprS e (ESM { esm_var = m_Var, esm_app = m_app }
  = case e of
     VarS x     -> Map.lookup x m_var
     AppS e1 e2 -> case lookupExprS e1 m_app of
                     Nothing -> Nothing
                     Just m1 -> lookupExprS e2 m1
\end{code}
This function pattern-matches on the target \lstinline{e}.  The \lstinline{LitS} alternative
says that to look up a literal, just look that literal up in the \lstinline{esm_lit} field.
But if the expression is an \lstinline{(AppS e1 e2)} node, we first look up \lstinline{e1}
in the \lstinline{esm_app} field, \emph{which returns a finite map}.  We then look up \lstinline{e2}
in that map.  Each distinct \lstinline{e1} yields a different map in which to look up \lstinline{e2}.

We can substantially abbreviate this code, at the expense of making it more cryptic, thus:
\begin{code}
lookupExprS (VarS x)     = esm_var >.> Map.Lookup x
lookupExprS (AppS e1 e2) = esm_app >.> lookupExprS e1 >=> lookupExprS e2
\end{code}
We use some simple composition combinators, whose types are given in \Cref{fig:library}
to chain together the component pieces.  The function \lstinline{esm_var :: ExprSMap v -> Map Var v}
is the auto-generated selector that picks \lstinline{esm_var} field from an \lstinline{ESM} record.
The functions \lstinline{>.>} and \lstinline{>=>} are forward composition
operators, respectively monadic and non-monadic, that chain the individual operations together.
Finally, we have eta-reduced the definition, by omitting the \lstinline{m} parameter.
These abbreviations become quite worthwhile when we add more constructors to the key data type.

Notice that in contrast to the approach of \Cref{sec:ord}, \emph{we never compare two expressions
for equality or ordering}.  We simply walk down the \lstinline{ExprMap} structure, guided
at each step by the next node in the target.  (We typically use the term ``target'' for the
key we are looking up in the finite map.)

This definition is extremely short and natural. But it conceals a hidden
complexity: \emph{they require polymorphic recursion}. The recursive call to \lstinline{lookupExprS e1}
instantiates \lstinline{v} to a differnet type than the parent function definition.
Haskell supports polymorphic recurision readily, provided you give type signature to
\lstinline{lookupExprS}, but not all languages do.

\subsection{Modifying tries} \label{sec:alter}

It is not enough to look up in a trie -- we need to \emph{build} them too!
Here is the code for \lstinline{alter}:
\begin{code}
alterExprS  :: ExprS -> XT v -> ExprSMap v -> ExprSMap v
alterExprS e xt m@(ESM { esm_var = m_var, esm_app = m_app }
  = case of
      VarS x     -> m { esm_var = Map.alter x xt m_var }
      AppS e1 e2 -> m { esm_app = alterExprS e1 (liftXT (alterExprS e2 xt)) m_app }

liftXT :: (ExprSMap v -> ExprSMap v) -> XT (ExprSMap)
liftXT f Nothing  = Just (f empExprS)
liftXT f (Just m) = Just (f m)
\end{code}
In the \lstinline{VarS} case, we must just update the map stored in the \lstinline{esm_var} field,
using the \lstinline{Map.alter} function from \Cref{fig:containers};
in Haskell the notation ``\lstinline|m { fld = e }|'' means the result
of updating the \lstinline{fld} field of record \lstinline{m} with new value \lstinline{e}.
In the \lstinline{AppS} case we look up \lstinline{e1} in \lstinline{m_app};
we should find a \lstinline{ExprSMap} there, which we want to alter with \lstinline{xt}.
We can do that with a recursive call to \lstinline{alterExprS}, using \lstinline{liftXT}
for impedence-matching.

The \lstinline{AppS} case shows why we need the generality of \lstinline{alter}.
Suppose we attempted to define an apparently-simpler \lstinline{insert} operations.
Its equation for \lstinline{(AppS e1 e2)} would look up \lstinline{e1} --- and would then
need to \emph{alter} that entry (an \lstinline{ExprSMap}, remember) with the result of
inserting \lstinline{(e2,v)}.  So we are forced to define \lstinline{alter} anyway.

We can abbreviate the code for \lstinline{alterExprS} using combinators, as we did in the case of
lookup, and doing so pays dividends when the key is a data type with
many constructors, each with many fields.  However, the details are
fiddly and not illuminating, so we omit them here.  Indeed, for the
same reason, in the rest of this paper we will typically omit the code
for \lstinline{alter}, though the full code is available in the
Appendix.

\subsection{Unions of maps}

A common operation on finite maps is to take their union:
\begin{code}
unionExprS :: ExprSMap v -> ExprSMap v -> ExprSMap v
\end{code}
In tree-based implementations of finite maps, such union operations can be tricky.
The two trees, which have been built independently, might not have the same
left-subtree/right-subtree structure, so some careful re-alignment may be requried.
But for tries there are no such worries --
their structure is identical, and we can simply zip them together:
\begin{code}
unionExprS (ESM { esm_var = m1_var, esm_app = m1_app })
           (ESM { esm_var = m2_var, esm_app = m2_app })
   = ESM { esm_var = Map.union m1_var m2_lt, esm_app = unionExprS m1_app m2_app }
\end{code}
It could hardly be simpler.

\subsection{Folds and the empty map} \label{sec:fold}

Of course, we need an empty trie. Here is one way to define such a thing:
\begin{code}
empExprS :: ExprSMap v
empExprS = ESM { esm_var = Map.empty, esm_app = empExprS }
\end{code}
It is interesting to note that \lstinline{empExprS} is an infinite, recursive structure:
the \lstinline{esm_app} field refers back to \lstinline{empExprS}.
This slightly strange definition works fine for lookup and alteration, but it fails
fundamentally when we want to iterate over the elements of the trie.

For example, suppose we wanted to count the number of elements in the finite map; in \lstinline{containers}
this is the function \lstinline{Map.size} (\Cref{fig:containers}).  We might try
\begin{code}
  sizeExprS :: ExprSMap v -> Int
  sizeExprS (ESM { esm_var = m_var, esm_app = m_app })
    = Map.size m_var + ???
\end{code}
We seem stuck because the size of the \lstinline{m_app} map is not what we want: rather,
we want to add up the sizes of its elements, and we don't have a way to do that yet.
The right thing to do is to generalise to a fold:
\begin{code}
  foldrExprS :: (v -> r -> r) -> r -> ExprSMap v -> r
  foldrExprS k z (ESM { esm_var = m_var, esm_app = m_app })
    = Map.foldr k z1 m_var
    where
      z1 = foldrExprS kapp z m_app
      kapp m1 r = foldrExprS k r m1
\end{code}
Here, in the binding for \lstinline{z1} we fold over \lstinline{m_app :: ExprSMap (ExprSMap v)}.
The function \lstinline{kapp} is combines the map we find with the accumulator, by again
folding over the map with \lstinline{foldrExprS}.

But alas, \lstinline{foldrExprS} will never terminate!  It always invokes itself immediately
(in \lstinline{z1}) on \lstinline{m_app}; but that invocation will again recursively invoke
\lstinline{foldrExprS}; and so on for ever.
The solution is simple: we just need an explicit representation of the empty map.
Here is one way to do it (we will see another in \Cref{sec:generalised}):
\begin{code}
data ExprSMap v = EmptyESM
                | ESM { esm_var :: Map Var v, esm_app :: ExprSMap (ExprSMap v) }

empExprS :: ExprSMap v
empExprS = EmptyESM

foldrExprS :: (v -> r -> r) -> r -> ExprSMap v -> r
foldrExprS k z EmptyESM                                   = z
foldrExprS k z (ESM { esm_var = m_var, esm_app = m_app }) = Map.foldr k z1 m_var
  where
    z1 = foldrExprS kapp z m_app
    kapp m1 r = foldrExprS k r m1
\end{code}
Equipped with a fold, we can easily define the size function, and another
that returns the range of the map:
\begin{code}
  sizeExprS :: ExprSMap v -> Int
  sizeExprS = foldrExprS (\_ n -> n+1) 0

  elemsExprS :: ExprSMap v -> [v]
  elemsExprS = foldrExprS (:) []
\end{code}

\subsection{Singleton maps} \label{sec:singleton}

Suppose we start with an empty map, and insert a value
with a key (an \lstinline{Expr}) that is large, say
\begin{code}
  App (App (Var "f") (Var "x")) (Var "y"))
\end{code}
Looking at the code
for \lstinline{alterExprS} in \Cref{sec:alter}, you can see that
because there is an \lstinline{App} at the root, we will build an
\lstinline{ESM} record with an empty \lstinline{esm_var}, and an
\lstinline{esm_app} field that is... another \lstinline{ESM}
record.  Again the \lstinline{esm_var} field will contain an
empty map, while the \lstinline{esm_app} field is a further \lstinline{ESM} record.

In effect, the key is linearised into a chain of \lstinline{ESM} records.
This is great when there are a lot of keys with shared structure, but
once we are in a sub-tree that represents a single key-value pair it is
a rather inefficient way to represent the key.  So a simple idea is this:
when a \lstinline{ExprMap} represents a single key-value pair, represent it
as directly a key-value pair!  Like this:
\begin{code}
data ExprSMap v = EmptyESM
                | SingleESM ExprS v
                | ESM { esm_var :: Map Var v, esm_app :: ExprSMap (ExprSMap v) }
\end{code}
The code for lookup practically writes itself:
\begin{code}
lookupExprS :: ExprS -> ExprSMap v -> Maybe v
lookupExprS e EmptyESM
  = Nothing
lookupExprS e1 (SingleESM e2 v2)
= if e1 == e2 then Just e2
              else Nothing
lookupExprS e (ESM { esm_var = m_var, esm_app = m_app }
  = ....     -- Exactly as before
\end{code}
Notice that in the \lstinline{SingleESM} case we need equality on \lstinline{Expr},
to tell if the key being looked up, \lstinline{k1} is the same as the key in
the \lstinline{SingleESM}, namely \lstinline{k2}.

The code for alter is more interesting, becuase it governs the shift from
\lstinline{EmptyESM} to \lstinline{SingleESM} and thence for \lstinline{ESM}:
\begin{code}
alterExprS  :: ExprS -> XT v -> ExprSMap v -> ExprSMap v
alterExprS e xt EmptyESM
  = case xt Nothing of
      Nothing -> EmptyESM
      Just v  -> SingleESM e v

alterExprS e xt m@(SingleESM key v1)
  | e == key
  = case xt (Just v1) of
      Nothing -> EmptyESM
      Just v2  -> SingleESM e v2
  | otherwise
  = case xt Nothing of
      Nothing -> m
      Just v2 -> alterExprS key (\_ -> Just v1) $
                 alterExprS e   (\_ -> Just v2) $
                 ESM { esm_var = Map.empty, esm_app = emptyESM }

alterExprS e xt m@(ESM { esm_var = m_var, esm_app = m_app }
  = case of
      Var x      -> Map.alter x xt m_var
      AppS e1 e2 -> alterExprS e1 (liftXT (alterExprS e2 xt)) m_app

liftXT :: (ExprSMap v -> ExprSMap v) -> XT (ExprSMap)
liftXT f Nothing  = Just (f empExprS)
liftXT f (Just m) = Just (f m)
\end{code}
Although we began by speaking of a map containing only one key-value pair,
this representation uses \lstinline{ESM} while there are keys that share structure,
but as soon as we get into a sub-treee where there is no overlap, we revert
to \lstinline{SingleESM}.

This optimisation makes a big difference in practice: see \Cref{sec:results}.

\subsection{Generalised singleton and empty maps} \label{sec:generalised}

Rather than implement the code for singleton maps and empty maps in every trie,
we can do it once and for all, like this:
\begin{code}
data SEMap m k v  -- Wrapper for singleton and empty map
  = EmptyTM | SingleTM k v | MultiTM (m v)

emptySEMap : SEMap m k v
emptySEMap = EmptyTM

lkSEMap :: Eq k => (k -> m v -> Maybe v) -> k -> SEMap m k v -> Maybe v
lkSEMap _  _  EmptyTM                    = Nothing
lkSEMap _  tk (SingleTM k v) | tk == k   = Just v
                             | otherwise = Nothing
lkSEMap lk tk (MultiTM m)                = lk tk m
\end{code}
Here \lstinline{lkSEMap} is responsible for the empty and singleton
cases, and delegates to the arugment function \lstinline{lk} in all other cases.
Now we can return to the simpler code in \Cref{sec:basic}, and define
\begin{code}
type GExprSMap v = SEMap ExprSMap ExprS v

lookupExprSMap :: ExprS -> GExprSMap v -> Maybe v
lookupExprSMap = lkSEMap lookupExprS
\end{code}
The code for \lstinline{xtSEMap}, and \lstinline{alterExprSMap}, follows straightforwardly.

\subsection{Maps of higher kinds}

Suppose our expresions had multi-argument apply nodes, thus
\begin{code}
data Expr = ...
          | AppV Expr [Expr]
\end{code}
Then we would need to built a trie keyed by a \emph{list} of \lstinline{Expr}.
Since a list is just another algebraic data type, build with nil and cons,
we can use exactly the same approach, thus
\begin{code}
lookupListExpr ::: [Expr] -> ListExprMap v -> Maybe v
\end{code}
But rather than build an implementation
for \lstinline{[Expr]}, and then another for \lstinline{[Decl]}, etc, we obviously
want to build a trie for lists of \emph{anything}, something like this \cite{hinze}:
\begin{code}
lookupList ::: [k] -> ListMap k v -> Maybe v
\end{code}
But this obviously cannot work: we need some type-class constraint on the key \lstinline{k},
saying that it can be used as the key of a trie.   That suggests
\begin{code}
lookupList :: TrieKey k => [k] -> TrieMap k v -> Maybe v

class Eq k => TrieKey k where
  type TrieMap k :: Type -> Type
  emptyTM  :: TrieMap k v
  lookupTM :: k -> TrieMap k v -> Maybe v
  alterTM  :: k -> XT v -> TrieMap k v -> TrieMap k v
  foldTM   :: (v -> r -> r) -> r -> TrieMap k v -> r
\end{code}
The class constraint \lstinline{TrieKey k} says that the type \lstinline{k}
can be used as the key of a triemap.
The class has an \emph{associated type}, \lstinline{TrieMap k},
a type-level function that transforms the type of the key into
the type of a trie for that key.  Now we can witness the fact that \lstinline{ExprS} can be
used as the key of a triemap, like this:
\begin{code}
instance TrieKey ExprS where
  type TrieMap ExprS = SEMap ExprSMap ExprS
  emptyTM  = emptySEMap
  lookupTM = lkSEMap lookupExprS
  ...
\end{code}
All this puts us in a position to write the instance for lists:
\begin{code}
instance TrieKey k => TrieKey [k] where
  type TrieMap [k] = SEMap (ListMap (TrieMap k)) [k]
  emptyTM  = emptySEMap
  lookupTM = lkSEMap lookupList
  ...

data ListMap elt_m v = LM { lm_nil  :: Maybe v, lm_cons :: elt_m (ListMap elt_m  v) }

lookupList :: TrieKey k => [k] -> TrieMap [k] v -> Maybe v
lookupList []     = lm_nil
lookupList (x:xs) = lm_app |> lookupTM y >=> lookupList ys
\end{code}
The code for \lstinline{alter} and \lstinline{fold} is routine.

\section{Keys with binders} \label{sec:binders}

Thus far we have usefully consolidated the state of the art, but have not really done
anything new.  Tries are well known, and there are a number of papers about
tries in Haskell \cite{hinze etc}.  However, none of these works deal with keys that contain
binders, and that should be insensitive to alpha-conversion.  That is the challenge we
address next.  Here is our data type
\begin{code}
data Expr = App Expr Expr | Lam Var Expr | Var Var
\end{code}
The key idea is simple: we perform de-Bruijn numbering on the fly,
renamign each binder to a natural number, from outside in.
So, when inserting or looking up a key $(\lambda x.\, foo~ (\lambda y.\, x+y))$ we
behave as if the key was $(\lambda.\, foo ~(\lambda. \bv{1} + \bv{2}))$, where
each $\bv{i}$ stands for an occurrence of the variable bound by the $i$'th lambda.
In effect, then, we behave as if the data type was like this:
\begin{code}
data Expr' = App Expr Expr | Lam Expr | FreeVar Var | BoundVar Int
\end{code}
Notice (a) the \lstinline{Lam} node no longer has a binder and (b) there are
two sorts of \lstinline{Var} nodes, one for free variables and one for bound
variables. We will not actually build a value of type \lstinline{Expr'} and look
that up in a trie keyed by \lstinline{Expr'}; rather,
we are going to \emph{behave as if we did}. Here is the code
\begin{code}
data ExprMap v = EM { em_app :: ExprMap (ExprMap v)
                    , em_lam :: ExprMap v
                    , em_fv  :: Map Var v           -- Free variables
                    , em_vb  :: Map BoundVarKey v } -- Lambda-bound variables

lookupExpr :: Expr -> ExprMap v -> Maybe v
lookupExpr e = lkExpr (DB emptyBVM e)

data DBExpr = DB { edb_bvm :: BoundVarMap, edb_expr = Expr }

lkExpr :: DBExpr -> ExprMap v -> Maybe v
lkExpr (DB bvm (App e1 e2)) = em_app >.> lkExpr (DB bvm e1) >=> lkExpr (DB bvm e2)
lkExpr (DB bvm (Lam v e))   = em_lam >.> lkExpr (DB (extendBVM v bvm) e)
lkExpr (DB bvm (Var v))     = case lookupBVM v bvm of
                                Nothing -> em_fv  >.> Map.lookup v  -- Free
                                Just bv -> em_bv  >.> Map.lookup v  -- Lambda-bound

data BoundVarMap = BVM { bvm_next :: BoundVarKey, bvm_map = Map Var BoundVarKey }
type BoundVarKey = Int

emptyBVM :: BoundVarMap
emptyBVM = BVM { bvm_next = 1, bvm_map = Map.empty }

extendBVM :: Var -> BoundVarMap -> BoundVarMap
extendBVM v (BVM { bvm_next = n, bvm_map = bvm })
  = BVM { bvm_next = n+1, bvm_map = Map.insert v n bvm }

lookupBVM :: Var -> BoundVarMap -> Maybe BoundVarKey
lookupBVM v (BVM {bvm_map = bvm }) = Map.lookup v bvm
\end{code}
We maintain a \lstinline{BoundVarMap}
that maps each lambda-bound variable to its de-Bruijn index, of type \lstinline{BoundVarKey}.  The key
we look up --- the first argument of \lstinline{lkExpr} --- becomes a \lstinline{DBExpr},
which is a pair of a \lstinline{BoundVarMap} and an \lstinline{Expr}.
At a \lstinline{Lam}
node we extend the \lstinline{BoundVarMap}. At a \lstinline{Var} node we
look up the variable in the \lstinline{BoundVarMap} to decide whether it is
lambda-bound (within the key) or free, and behave appropriately.
The code for \lstinline{alter} and \lstinline{fold} holds no new surprises.
The construction of \Cref{sec:generalised}, to handle empty and singleton maps,
applies without difficulty to this generalised map.

And that is really all there is to it.  We regard this as a non-obvious merit
of the entire trie approach: it is quite remarkably easy to extend the basic
trie idea to be insensitive to alpha-conversion.

\section{Tries that match}

Next, we extend our tries to accomodate \emph{matching}, as we
sketched in \Cref{sec:matching-intro}.  A key advantage of tries over other representations is
that they can naturally extend to support matching.

\subsection{What ``matching'' means} \label{sec:matching-spec}

First, we have to ask what the API should be.
Our overall goal is to build a \emph{matching trie} into which we can:
\begin{itemize}
\item \emph{Insert} (pattern, value) pairs
\item \emph{Look up} a target expression, and return all the values whose pattern \emph{matches} that expression.
\end{itemize}
Semantically, then, a matching trie can be thought of as a set of (pattern, value) pairs.
What is a pattern? It is a pair $(vs,p)$ where
\begin{itemize}
\item $vs$ is a set of \emph{pattern variables}, such as $[a,b,c]$.
\item $p$ is a \emph{pattern expression}, such as $f\, a\, (g\, b\, c)$.
\end{itemize}
A pattern may of course contain free variables (not bound by the pattern), such as $f$ and $g$
in the above example.
A pattern $(vs, p)$ \emph{matches} a target expression $e$ iff there is a unique substitution
$S$ whose domain is $vs$, such that $S(p) = e$.

We allow the same variable to occur more than once in the pattern.
For example, suppose we wanted to encode the rewrite rule
\begin{code}
{#- RULE "foo" forall x. f x x = f2 x #-}
\end{code}
Here the pattern $([x], f~ x~ x)$ has a repeated variable $x$,
and should match targets like $(f~ 1~ 1)$ or $(f ~(g~ v)~ (g ~v))$,
but not $(f~ 1~ (g~ v))$.  This ability important if we are to use matching tries
to implement class or type-family look in GHC.

It is sometimes desirable to be able to look up the \emph{most specific match} in the matching trie.  For example, suppose the matching trie contains
$$
\{ ([a],\, f\, a), ([p,q],\, f\,(p+q)) \}
$$
and suppose we look up $(f\,(2+x))$ in the trie.  The first entry matches, but the second also matches (with $S = [p \mapsto 2, q \mapsto x]$), and \emph{the second pattern is a substitution instance of the first}.  So we may want to return just the second match.  We call this \emph{most-specific matching}.

\subsection{The API of a matching trie} \label{sec:match-api}

Here are the signatures of the lookup and insertion\footnote{We begin with \lstinline{insert}
  because it is simpler than \lstinline{alter}} functions:
\begin{code}
type ExprPat = ([PatVar], Expr)
type PatVar  = Var
type Match v = ([(PatVar, Expr)], v)

insertMExpr :: ExprPat -> v -> MExprMap v -> MExprMap v
lookupMExpr :: Expr -> MExprMap v -> Bag (Match v)
\end{code}
A \lstinline{MExprMap} is a trie, keyed by \lstinline{Expr} \emph{patterns}.
A pattern variable, of type \lstinline{PatVar} is just a \lstinline{Var}; we
use the type synonym just for documentation purposes. When inserting into a
\lstinline{MExprMap} we supply a pattern expression paired with the \lstinline{[PatVar]}
over which the pattern is quantified.  When looking up in the map we return a \emph{bag}
of results (because more than one pattern might match).  Each item in this bag is
a \lstinline{Match} that includes the \lstinline{(PatVar, Expr)} pairs obtained by
matching the pattern, plus the value in the map (which presumably mentions those
pattern variables).

\subsection{Canonical patterns and pattern keys}

In \Cref{sec:binders} we saw how we could use de-Bruijn numbers to
make two lambda expressions that differ only superficially (in the
name of their bound variable) look the same.  Clearly, we want to do
the same for pattern variables.  After all, consider these two patterns:
$$
([a,b], f~a~b~True) \qquad and \qquad ([p,q], f~q~p~False)
$$
The two pattern expressions share a common prefix, but differ both in the
\emph{names} of the pattern variable and in their \emph{order}. We might hope
to suppress the accidental difference of names by using numbers instead -- we will
use the term \emph{pattern keys} for these numbers.
But from the set of pattern variables alone, we
cannot know \emph{a priori} which key to assign to which variable.

Our solution is to number the pattern variables \emph{in order of their
first occurrence in a left-to-right scan of the expression}\footnote{As we shall
  see, this is very convenient in implementation terms.}.
As in \Cref{sec:binders} we will imagine that we cannicalise the pattern, although
in reality we will do so on-the-fly, without ever constructing the cannonicalised pattern.
Be that as it may, the canonicalised patterns become:
$$
   f~\pv{1}~\pv{2}~True      \qquad and \qquad  f~\pv{1}~\pv{2}~False
$$
By numbering the variables left-to-right, we ensure that they ``line up''.
In fact, since the pattern variables are numbered left-to-right we don't even
need the subscripts (just as we don't need a subscript on the lambda in
de-Bruijn notation), so the canonicalised patterns become
$$
   f~\pv{}~\pv{}~True      \qquad and \qquad  f~\pv{}~\pv{}~False
$$


What if the variable occurs more than once? For example, suppose we are matching
the pattern $([x],\, f\, x\,x\,x)$ against the target expression
$(f\,e_1\,e_2\,e_3)$.  At the first occurrence of the pattern variable $x$
we succeed in matching, binding $x$ to $e_1$; but at the second
occurrence we must note that $x$ has already been bound, and instead
check that $e_1$ is equal to $e_2$; and similarly at the third occurrence.
These are very different actions,
so it is helpful to distinguish the first occurrence from subsequent
ones when canonicalising.  So our pattern $([x],\, f\, x\,x\,x)$ might
be canonicalised to $(f\,\pv{}\,\pvo{1}\,\pvo{1})$, where the first (or binding) occurrence
is denoted $\pv{}$ and subsequent (bound) occurrences of pattern variable $i$ are denoted $\pvo{i}$.

For pattern-variable occurrences we really do need the subcript! Consider the
patterns $$([x,y], f\,x\,y\,y\,x) \qquad and \qquad ([p,q], f\,q\,p\,q\,p)$$
which differ not only in the names of their pattern variables, but also in the
order in which they occur in the pattern.
They canonicalise to
$$(f \,\pv{}\, \pv{}\, \pvo{2}\, \pvo{1}) \qquad and  \qquad (f \,\pv{}\, \pv{}\, \pvo{1}\, \pvo{2})$$
respectively.  The subscripts are essential to keep these two patterns distinct.

\subsection{Undoing the pattern keys} \label{sec:patkeymap}

The trouble with canonicalising our patterns (to share the structure of the patterns)
is that matching will produce a substitution mapping patttern \emph{keys} to
expressions, rather that mapping pattern \emph{variables} to values.  For example,
suppose we start with the pattern $([x,y], f \,x\, y\, y\, x)$ from the
end of the last section. Its canonical form is $(f \,\pv{}\, \pv{}\, \pvo{2}\, \pvo{1})$.
If we match that against a target $(f\,e_1\,e_2\,e_2\,e_1)$ we will produce a substitution $[\pvo{1} \mapsto e_1, \pvo{2} \mapsto e_2]$.
But  what we \emph{want} is a \lstinline{Match} (\Cref{sec:match-api}),
that gives a list of (pattern-variable, value) pairs $[(x, e_1), (y,e_2)]$.

Somehow we must accumulate a \emph{pattern-key map} that, for each
individual entry, maps its pattern keys back to its corresponding
pattern variables.  The pattern-key map is just a list of (pattern-variable, pattern-key) pairs.
For our example the pattern key map would be
$[(x, \pv{1}), (y,\pv{2})]$.  We can store the pattern key
map paired with the value, so that once we find a successful match we can use the pattern
key map and the pattern-key substitution to recover the pattern-variable substition that we want.

To summrise, suppose we want to build a matching trie for the following (pattern, value) pairs:
$$
(([x,y],\; f\;y\;(g\;y\;x)),\; v_1) \qquad and \qquad (([a],\; f\;a\;True),\;v_2)
$$
Then we will build a trie for the followng key-value pairs
$$
( (f \;\pv{}\;(g\;\pvo{1}\;\pv{})),\; (([(x,\pv{2}),(y,\pv{1})]), v_1) )
  \qquad and \qquad
( (f \;\pv{}\;True),\; ([(a,\pv{1})],\;v_2) )
$$


\subsection{Implementation: lookup}

We are finally ready to give an implementation of matching tries.
We begin with \lstinline{ExprS} (defined in \Cref{sec:ExprS}) as our key type;
that is we will not deal with lambdas and lambda-bound variables for now.
\Cref{sec:binders} will apply with no difficulty, but we can add that back
in after we have dealt with matching.

With these thoughts in mind, our matching trie has this definition:
\begin{code}
type PatKeys     = [(PatVar,PatKey)]
type MExprSMap v = MExprSMapX (PatKeys, v)

data MExprSMapX v
    = MM { mm_app  :: MExprSMap (MExprSMap v)
         , mm_fvar :: Map Var v
         , mm_pvar :: Maybe v     -- First occurrence of a pattern var
         , mm_xvar :: PatOccs v   -- Subsequent occurrence of a pattern var
       }
type PatOccs v = [(PatKey,v)]
\end{code}
The client-visible \lstinline{MExprSMap} with values of type \lstinline{v}
is a matching trie \lstinline{MExprSMapX} with values of type \lstinline{(PatKeys,v)},
as described in \Cref{sec:patkeymap}.
The trie \lstinline{MExprSMapX} has four fields, one for each case in the pattern.
The first two fields deal with literals and applications, just as before. The third deals with the \emph{binding} occurrence
of a pattern variable $\pv{}$, and the fourth with a \emph{bound} occurrence of
a pattern variable $\pvo{i}$.

The core lookup function looks like this:
\begin{code}
lkMExprS :: forall v. ExprS -> (PatSubst, MExprSMapX v) -> Bag (PatSubst, v)

type PatKey = Int
data PatSubst = PS { ps_next  :: PatKey, ps_subst :: Map PatKey ExprS }
\end{code}
As well as the target expression \lstinline{ExprS} and the trie, the lookup function also takes
a \lstinline{PatSubst} that gives the bindings for pattern variable bound so far.
It returns a bag of results, since more than one entry in the trie may match,
each paired with the \lstinline{PatSubst} that binds the pattern variables.

Given \lstinline{lkMExprS} we can write \lstinline{lookupMExpr},
the externally-callable lookup function:
\begin{code}
lookupMExprS :: Expr -> MExprSMap v -> Bag (Match v)
lookupMExprS e m = map rejig (lkMExprS e m)
  where
    rejig :: (PatSubst, (PatKeys, v)) -> Match v
    rejig (ps, (pkmk, v)) = (map (lookupPatKey ps) pmk, v)

lookupPatKey :: PatSubst -> (PatVar,PatKey) -> (Papvar,ExprS)
lookupPatKey subst (pat_var, pat_key) = (pat_var, lookupPatSubst pat_key subst)

lookupPatSubst :: PatKey -> PatSubst -> ExprS
lookupPatSubst pat_key (PS { ps_subst = subst })
  = case Map.lookup pat_key pat_subst of
      Just expr -> expr
      Nothing   -> error "Unbound key"
\end{code}
Here \lstinline{lookupMExpr} is just an impedence-matching shim around
a call to \lstinline{lkMExprS} that does all the work.  Notice that the
input.  The latter returns a bag of \lstinline{(PatSubst, (PatKeys, v))}
values, which the function \lstinline{rejig} converts into the
the \lstinline{Match v} results that we want.  The ``unbound key''
failure case in \lstinline{lookupPatSubst} means that
\lstinline{PatKeys} in a looked-up value asks for a key that is not
bound in the pattern.  The insertion function will ensure that this
never occurs.

Now we can return to the recursive function that does all the work: \lstinline{lkMExprS}:
\begin{code}
lkMExprS :: forall v. ExprS -> (PatSubst, MExprSMapX v) -> Bag (PatSubst, v)
lkMExprS e (psubst, mt)
  = pat_var_bndr `Bag.union` pat_var_occs `Bag.union` look_at_e
  where
     pat_var_bndr :: Bag (PatSubst, v)
     pat_var_bndr = case mm_pvar of
                      Just e  -> Bag.single (extendPatSubst e tsubst, x)
                      Nothing -> Bag.empty

     pat_var_occs :: Bag (PatSubst, v)
     pat_var_occs = fromList [ (psubst, v)
                             | (pat_var, v) <- mm_xvar
                             , e == lookupPatSubst pat_var psubst ]

     look_at_e :: Bag (PatSubst, v)
     look_at_e = case e of
        Var x     -> case Map.lookup lit (mm_var mt) of
                       Just v  -> Bag.single (psubst,v)
                       Nothing -> Bag.empty
        App e1 e2 -> (psubst, m_app mt) |> lkMExprS e1 >=> lkMExprS e2
\end{code}
The bag of results is the union of three possibilities, as follows. (Keep in mind that a \lstinline{MExprSMap} represents \emph{many} patterns simultaneously.)
\begin{itemize}
\item \lstinline{pat_var_bndr}: we consult the \lstinline{mm_pvar}, if it contains \lstinline{Just v} then at least one of the patterns in this trie has a pattern binder $\pv{}$ at this spot.  In that case we can simply bind the next free pattern variable (\lstinline{ps_next}) to \lstinline{e}, and return a singleton bag.
\item \lstinline{pat_var_occs}: any of the bound pattern varaibles might have an occurrence $\pvo{i}$ at this spot, and a list of such bindings is held in \lstinline{pat_var_occs}.  For each, we must do an equality check between the target \lstinline{e} and the expression bound to that pattern variable (found via \lstinline{lookupPatSubst}).  We return a bag of all values for which the equality check succeeds.
  \item \lstinline{look_at_e} corresponds exactly to the cases we saw before in \Cref{sec:ExprS}.   The only subtlety is that we are are returning a \emph{bag} of results, but happily the Kleisli composition operator \lstinline{(>=>)} (\Cref{fig:library}) works for any monad, including bags.
\end{itemize}

\subsection{Altering a matching trie}

\simon{Too much code, I know; but this section is one of the key contributions of the paper.}

How did the entries in our map get their \lstinline{PatKeys}?  That
is, of course, the business of \lstinline{insert}, or more generally
\lstinline{alter}.  The key, recursive function must carry inwards a mapping
from pattern variables to pattern keys; we can simply re-use \lstinline{BoundVarMap}
from \Cref{sec:bvm} for this purpose.  The exact signature for the function takes
a bit of puzzling out, and is worth comparing with its predecessor in \Cref{sec:alter}:
\begin{code}
type PatKeyMap = BoundVarMap   -- We re-use BoundVarMap

xtMExprS :: Set PatVar -> ExprS -> (PatKeyMap -> XT a)
         -> PatKeyMap -> MExprSMapX v -> MExprSMapX v
\end{code}
It is unsurprising the the function is given the set of pattern variables, so that it
can distinguish pattern variables from free variables.  It also takes a \lstinline{PatKeyMap}, the
current binding of already-encountered pattern variables to their pattern keys;
when it completes the lookup it passes that completed binding map to the ``alter'' function.

Given this workhorse, we can build the client-visible \lstinline{insert} function\footnote{\lstinline{alter} is not much harder.}:
\begin{code}
insertExprS :: forall v. [Var]     -- Pattern variables
                         -> ExprS  -- Pattern
                         -> v -> MExprSMap v -> MExprSMap v
insertExprS pat_vs e v mm
  = xtExprS (Set.fromList pat_vs) e xt emptyBVM mm
  where
    xt :: PatKeyMap -> XT (PatKeys, v)
    xt pkm _ = Just (map inst_key pat_vs, v)
     -- The "_" means just overwrite previous value
     where
        inst_key :: PatVar -> (PatVar, PatKey)
        inst_key x = case lookupBVM x pkm of
                         Nothing -> error ("Unbound pattern variable " ++ x)
                         Just pk -> (x, pk)
\end{code}
This is the code that builds the \lstinline{PatKeys} in the range of the map.
It does so using the \lstinline{PatKeyMap} accumulated by \lstinline{xtExprS} and
finally passed to the local function \lstinline{xt}.

Now we can define the workhorse, \lstinline{xtExprS}:
\begin{code}
xtExprS pvs e xt pkm mm
  = case e of
      AppS e1 e2 -> mm { mm_app = xtMExprS pvs e2 (liftXT (xtMExprS pvs e2 f))
                                           pkm (mm_app mm) }

      VarS x | Just xv <- lookupBVM tv pkm
             -> -- Second or subsequent occurrence of a pattern variable
                mm { tm_xvar = xtPatVarOcc xv (f pkm) (tm_xvar mm) }

             | pv `Set.member` pvs
             -> -- First occurrence of a pattern variable
                mm { tm_tvar = xt (extendBVM pv pkm) (tm_tvar mm) }

             | otherwise
             -> -- A free variable
                mm { tm_fvar = Map.alter (xt pmm) x (tm_fvar mm) }

liftXT :: (PatKeys -> MExprSMap v -> MExprSMap v)
        -> PatKeys -> Maybe (MExprSMap v) -> Maybe (MExprSMap v)
liftXT insert pkeys Nothing  = Just (insert pkeys emptyMExprSMapX)
liftXT insert pkeys (Just m) = Just (insert pkeys m)

xtPatVarOcc :: PatKey -> XT v -> TmplOccs v -> TmplOccs v
xtPatVarOcc key f []
  = xtCons key (f Nothing) []
xtPatVarOcc key f ((key1,x):prs)
  | key == key1 = xtCons key (f (Just x)) prs
  | otherwise   = (key1,x) : xtPatVarOcc key f prs

xtCons :: PatKey -> Maybe a -> PatOccs a -> PatOccs a
xtCons _   Nothing  pat_occs = pat_occs
xtCons key (Just x) pat_occs = (key,x) : pat_occs
\end{code}

\subsection{Most specific match}

\Cref{sec:matching-spec} described the goal of returning only the \emph{most specific matches} from
a lookup.  In GHC today, the lookup returns \emph{all} matches, and these matches are then
exhaustively compared against each other; if one is more specific than (a substitution instance of) another, the latter is discarded.

A happy consequence of the trie representation is that a one-line change suffices
to return only the most-specific matches.  We simply modify the definition of
\lstinline{pat_var_bndr} in \lstinline{lkMExprS}, we simply test for 
\begin{code}
lkMExprS e (psubst, mt)
  = pat_var_bndr `Bag.union` pat_var_occs `Bag.union` look_at_e
\end{code}

\section{Related work}

* Using a FSM; e.g \emph{Interpreted Pattern Match Execution} by Jeff Niu, a UW undergrad intern at Google.  https://docs.google.com/presentation/d/1e8MlXOBgO04kdoBoKTErvaPLY74vUaVoEMINm8NYDds/edit?usp=sharing

* Matching multiple strings.

\end{document}


