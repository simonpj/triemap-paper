% Links
% https://stackoverflow.com/questions/16084788/generic-trie-haskell-implementation
% Hinze paper: http://citeseerx.ist.psu.edu/viewdoc/summary?doi=10.1.1.8.4069

%% For double-blind review submission, w/o CCS and ACM Reference (max submission space)
% \documentclass[acmsmall,review]{acmart}\settopmatter{printfolios=true,printccs=false,printacmref=false}
%% For double-blind review submission, w/ CCS and ACM Reference
%\documentclass[acmsmall,review,anonymous]{acmart}\settopmatter{printfolios=true}
%% For single-blind review submission, w/o CCS and ACM Reference (max submission space)
% \documentclass[acmsmall,review]{acmart}\settopmatter{printfolios=true,printccs=false,printacmref=false}
%% For single-blind review submission, w/ CCS and ACM Reference
%\documentclass[acmsmall,review]{acmart}\settopmatter{printfolios=true}
%% For final camera-ready submission, w/ required CCS and ACM Reference
\documentclass[format=sigplan]{acmart}

%% Journal information
%% Supplied to authors by publisher for camera-ready submission;
%% use defaults for review submission.
% \acmJournal{PACMPL}
% \acmVolume{1}
% \acmNumber{ICFP} % CONF = POPL or ICFP or OOPSLA
% \acmArticle{1}
% \acmYear{2020}
% \acmMonth{8}
% \acmDOI{} % \acmDOI{10.1145/nnnnnnn.nnnnnnn}
% \startPage{1}

%% Copyright information
%% Supplied to authors (based on authors' rights management selection;
%% see authors.acm.org) by publisher for camera-ready submission;
%% use 'none' for review submission.
% \setcopyright{none}
%\setcopyright{acmcopyright}
%\setcopyright{acmlicensed}
%\setcopyright{rightsretained}
%\copyrightyear{2018}           %% If different from \acmYear

%%% The following is specific to ICFP '20 and the paper
%%% 'Kinds Are Calling Conventions'
%%% by Paul Downen, Zena M. Ariola, Simon Peyton Jones, and Richard A. Eisenberg.
%%%
\setcopyright{rightsretained}
\acmPrice{}
\acmDOI{10.1145/3408986}
\acmYear{2020}
\copyrightyear{2020}
\acmSubmissionID{icfp20main-p93-p}
\acmJournal{PACMPL}
\acmVolume{4}
\acmNumber{ICFP}
\acmArticle{104}
\acmMonth{8}

%% Bibliography style
\bibliographystyle{ACM-Reference-Format}
%% Citation style
%% Note: author/year citations are required for papers published as an
%% issue of PACMPL.
\citestyle{acmauthoryear}   %% For author/year citations


%%%%%%%%%%%%%%%%%%%%%%%%%%%%%%%%%%%%%%%%%%%%%%%%%%%%%%%%%%%%%%%%%%%%%%
%% Note: Authors migrating a paper from PACMPL format to traditional
%% SIGPLAN proceedings format must update the '\documentclass' and
%% topmatter commands above; see 'acmart-sigplanproc-template.tex'.
%%%%%%%%%%%%%%%%%%%%%%%%%%%%%%%%%%%%%%%%%%%%%%%%%%%%%%%%%%%%%%%%%%%%%%


%% Some recommended packages.
\usepackage{booktabs}   %% For formal tables:
                        %% http://ctan.org/pkg/booktabs
\usepackage{subcaption} %% For complex figures with subfigures/subcaptions
                        %% http://ctan.org/pkg/subcaption
\usepackage[utf8]{inputenc}
\usepackage{microtype}
\usepackage{amsmath}
\usepackage{amsthm}
% \usepackage{amssymb}
% \usepackage{stmaryrd}
\usepackage{framed}
\usepackage{proof}
\usepackage{braket}
\usepackage{fancyvrb}
\usepackage{listings}
\usepackage[inline,shortlabels]{enumitem}
\usepackage[capitalize]{cleveref}
\usepackage{xcolor}
\usepackage{pgffor}
\usepackage{ragged2e}

% \RequirePackage{xargs}

\VerbatimFootnotes

\lstset{language=Haskell}

\let\restriction\relax

\theoremstyle{theorem}
\newtheorem{proposition}{Proposition}
\newtheorem{lemma}{Lemma}
\newtheorem{theorem}{Theorem}
\newtheorem{corollary}{Corollary}
\newtheorem{property}{Property}
\theoremstyle{definition}
\newtheorem{definition}{Definition}
\newtheorem{restriction}{Restriction}
\newtheorem{intuition}{Intuition}
\theoremstyle{remark}
\newtheorem{remark}{Remark}
\newtheorem{notation}{Notation}

%% Style guide forbids boxes around figures
% \newenvironment{figurebox}{\begin{figure}\begin{framed}}{\end{framed}\end{figure}}
% \newenvironment{figurebox*}{\begin{figure*}\begin{framed}}{\end{framed}\end{figure*}}
\newenvironment{figurebox}{\begin{figure}}{\end{figure}}
\newenvironment{figurebox*}{\begin{figure*}}{\end{figure*}}

\crefname{figure}{Fig.}{Figs.}
\Crefname{figure}{Fig.}{Figs.}
\crefname{restriction}{Restriction}{Restrictions}

% Some colors:
\definecolor{dkcyan}{rgb}{0.1, 0.3, 0.3}
\definecolor{dkgreen}{rgb}{0,0.3,0}
\definecolor{olive}{rgb}{0.5, 0.5, 0.0}
\definecolor{dkblue}{rgb}{0,0.1,0.5}

\definecolor{col:ln}{rgb}  {0.1, 0.1, 0.7}
\definecolor{col:str}{rgb} {0.8, 0.0, 0.0}
\definecolor{col:db}{rgb}  {0.9, 0.5, 0.0}
\definecolor{col:ours}{rgb}{0.0, 0.7, 0.0}

\definecolor{lightgreen}{RGB}{170, 255, 220}
\definecolor{darkbrown}{RGB}{121,37,0}

% Customized syntax highlighting for Haskell code snippets:
\colorlet{listing-comment}{gray}
\colorlet{operator-color}{darkbrown}

\lstdefinestyle{default}{
    basicstyle=\ttfamily\fontsize{8.7}{9.5}\selectfont,
    columns=fullflexible,
    commentstyle=\sffamily\color{black!50!white},
    escapechar=\#,
    framexleftmargin=1em,
    framexrightmargin=1ex,
    keepspaces=true,
    keywordstyle=\color{dkblue},
    mathescape,
    numbers=none,
    numberblanklines=false,
    numbersep=1.25em,
    numberstyle=\relscale{0.8}\color{gray}\ttfamily,
    showstringspaces=true,
    stepnumber=1,
    xleftmargin=1em
}

\lstdefinelanguage{custom-haskell}{
    language=Haskell,
    deletekeywords={lookup, delete, map, mapMaybe, Ord, Maybe, String, Just, Nothing, Int, Bool},
    keywordstyle=[2]\color{dkgreen},
    morekeywords=[2]{String, Map, Ord, Maybe, Int, Bool},
    morekeywords=[2]{Name, Expression, ESummary, PosTree, Structure, HashCode, VarMap},
    keywordstyle=[3]\color{dkcyan},
    literate=%
        {=}{{{\color{operator-color}=}}}1
        {|}{{{\color{operator-color}|}}}1
        {\\}{{{\color{operator-color}\textbackslash$\,\!$}}}1
        {.}{{{\color{operator-color}.}}}1
        {=>}{{{\color{operator-color}=>}}}1
        {->}{{{\color{operator-color}->}}}1
        {<-}{{{\color{operator-color}<-}}}1
        {::}{{{\color{operator-color}::}}}1
}

\lstset{style=default}
% Environment for code snippets
\lstnewenvironment{code}[1][]
  {\small\lstset{language=custom-haskell,#1}}
  {}

% Environment for example expressions
\lstnewenvironment{expression}[1][]
  {\small\lstset{#1}}
  {}

\begin{document}

\newcommand{\simon}[1]{{\bf SLPJ}: #1 {\bf End SLPJ}}


%% Title information
\title%[Short Title]
{Triemaps that match}         %% [Short Title] is optional;
                                        %% when present, will be used in
                                        %% header instead of Full Title.
% \titlenote{with title note}             %% \titlenote is optional;
%                                         %% can be repeated if necessary;
%                                         %% contents suppressed with 'anonymous'
% \subtitle{Subtitle}                     %% \subtitle is optional
% \subtitlenote{with subtitle note}       %% \subtitlenote is optional;
%                                         %% can be repeated if necessary;
%                                         %% contents suppressed with 'anonymous'


%% Author information
%% Contents and number of authors suppressed with 'anonymous'.
%% Each author should be introduced by \author, followed by
%% \authornote (optional), \orcid (optional), \affiliation, and
%% \email.
%% An author may have multiple affiliations and/or emails; repeat the
%% appropriate command.
%% Many elements are not rendered, but should be provided for metadata
%% extraction tools.

%% Author with single affiliation.
\author{Simon Peyton Jones}
\affiliation{
  \institution{Microsoft Research}
  % \streetaddress{21 Station Rd.}
  \city{Cambridge}
  % \postcode{CB1 2FB}
  \country{UK}
}
\email{simonpj@microsoft.com}

\author{Richard A.~Eisenberg}
\affiliation{
  \institution{Tweag I/O}
  \city{Cambridge}
  \country{UK}
}
\email{rae@richarde.dev}

\author{Josef Sveningsson}

%% Abstract
%% Note: \begin{abstract}...\end{abstract} environment must come
%% before \maketitle command
\begin{abstract}
TrieMaps are great.
\end{abstract}

%% \maketitle
%% Note: \maketitle command must come after title commands, author
%% commands, abstract environment, Computing Classification System
%% environment and commands, and keywords command.
\maketitle


\section{Introduction} \label{sec:intro}

The designs of many programming languages include a feature where some concrete use of a construct is matched against a set of possible interpretations, where the possible interpretations might be defined in terms of variables to be instantiated at concrete usages. For example:
\begin{itemize}
\item Haskell's class instances work this way: the user defines instances (which may contain variables), and then concrete usage sites of class methods require finding an instance that applies.
\item Agda, Coq, and Idris all have implicit-argument features directly inspired by Haskell's class mechanism.
\item An extension to Haskell allows overloaded instances, where we select the most specific instance (that is, one that is a specialization of any other possible matching instance).
\item C++ and Java both support function overloading, where a usage site of a function must be matched against a choice of implementation. C++'s templates and Java's generics allow for variables to be used in the implementations. There are sometimes multiple implementations that match; both languages choose the most specific match (for an appropriate definition of "most specific").
\item C++ separately allows template specialization, where a templated definition can have concrete specializations. Once again, selection of these specializations depends on a "most-specific" relation appropriate to the case.
\item Scala? C\#?
\end{itemize}
Beyond features specified in a language's design, optimizations may require such a structure. For example, GHC's rewrite rules~\cite{rewrite-rules} requires a similar lookup to find a mapping from expressions to rules that may apply.

Our concern here is the efficient implementation of this matching
operation. That is, we wish to define a structure mapping keys to
arbitrary values. The keys are chosen from a type described by a
context-free grammar and thus comprise small trees. We assume a total
ordering on such trees. The challenge lies in the fact that we want
our structure to support wildcard variables, which represent any tree
at all. We will write such variables with Greek letters. Accordingly,
we can view a mapping \lstinline{alpha |-> v} as an infinite mapping, connecting
all possible trees to v, or we can have \lstinline{Maybe alpha |-> v2} map all
trees whose root is Maybe to the value v2. Combining these to `alpha
|-> v, Maybe alpha |-> v2` would map all trees to v, except those
trees that have Maybe at the root, which map to v2. Accordingly,
looking up in our structure find the most specific
match.\footnote{This aspect of our design is unforced. We could also
return all possible matches, instead of selecting the most
specific. See \cref{sec:most-specific}.}

Our contributions are as follows:

\begin{itemize}
\item We describe the language-agnostic Variable Trie-Map (VTM) data structure, with the semantics as described above (and made more precise in \cref{sec:vtm}). The keys in a VTM support local bound variables via on-the-fly conversion to de Bruijn levels, necessary to support polymorphic types or $\lambda$-expressions. Looking up in a VTM is linear in the size of the key, regardless of the size of the VTM itself. The operations on a VTM are proved to uphold the sensible properties described in \cref{sec:vtm-properties}.
\item Some languages require not only the matching behavior above, but also a \emph{unification} operation, where we find not only keys that match, but keys that would match if variables in the looked-up tree were further instantiated. \Cref{sec:unification} describes how we extend VTMs to support unification as well as matching.
\item While VTMs are time-efficient, they can be space-inefficient. \Cref{sec:path-compression} describes an easy optimization which drastically reduces the memory footprint of VTMs whose keys share little structure in common. This optimization shows a 30% geometric mean savings in the size of the structure GHC uses to look up rewrite rules.
\item Our work was motivated by quadratic behavior in GHC, observed when checking for instance consistency among imports. This became a problem in practice in Haskell's use at Facebook. We report on our implementation within GHC, showing that it achieves a 98% speedup against the previous (admittedly naive) implementation of instance tables.
\end{itemize}

\section{Job description}

Our general task is as follows: \emph{implement a finite mapping from keys to values,
in which the key type is some kind of tree}.
For example, a \lstinline|TypeMap a| might map a \lstinline{Type} to a value of type \lstinline{a}, providing operations
such as\footnote{The function \lstinline{alterTM} is a standard generalisation of \lstinline{insert},
  in which the element to be inserted is generalised to a function \lstinline{Maybe a -> Maybe a}.
  This function transforms the existing value associated with key, if any (hence the \lstinline{Maybe}), to
  a new value, with \lstinline{Nothing} indicating deletion.}
\begin{code}
  emptyTM  :: TypeMap a
  lookupTM :: Type -> TypeMap a -> Maybe a
  insertTM :: Type -> a -> TypeMap a -> TypeMap a
  alterTM  :: Type -> (Maybe a -> Maybe a)
              -> TypeMap a -> TypeMap a
\end{code}
Here a \lstinline{Type} might be defined like this:
\begin{code}
  data Type = TyConTy TyCon
            | FunTy Type Type
            | ForAllTy TyVar Type
            | TyVarTy TyVar
\end{code}
This data type is capable of representing types like $Int \rightarrow
Int$ or $\forall a. a \rightarrow Int \rightarrow a$.  Notice that we
would expect insertion and lookup to be insensitive to
$\alpha$-renaming. For example suppose we insert a value with key
$((\forall a. a \rightarrow a) \rightarrow Int)$, and then look up the
key $((\forall b. b \rightarrow b) \rightarrow Int)$: we should
certainly expect to find the value we inserted, even though the name
of the $\forall$-bound variable has changed.

In GHC we want more: we want to a lookup that does \emph{matching}.
\simon{Say more}.

\section{Non-solutions}

At first sight the job can be done easily: define a total order on
\lstinline{Type}, make it an instance of the class \lstinline{Ord},
and use a standard finite map library such as \lstinline{Data.Map}.
And indeed that works, but it is terribly slow.  A finite map is
implemented as a binary search tree; at every node of this tree, we compare the key (a \lstinline{Type}, remember) with
a key stored at the node; if it is smaller, go left; if larger, go right. So each lookup
a (logarithmic) number of potentially-full-depth comparisons of two types.

Another possibility might be to hash the \lstinline{Type} and use the
hash-code as the lookup key.  That would make lookup much faster, but
it requires at least two full traversals of the key for every lookup:
one to compute its hash code for every lookup, and a full equality
comparison on a ``hit'' because hash-codes can collide.  Moreover
hashing algorithms that respect $\alpha$-conversion are not trivial
\cite{alpha-hashing}.

\section{Tries}

A rather standard approach to a finite map in which the key has internal structure
is to use a \emph{trie}.

\end{document}
