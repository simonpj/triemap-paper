\documentclass[sigplan,dvipsnames]{acmart}

\acmJournal{PACMPL}
\acmVolume{1}
\acmNumber{POPL} % CONF = POPL or ICFP or OOPSLA
\acmArticle{1}
\acmYear{2022}
\acmMonth{1}
\acmDOI{} % \acmDOI{10.1145/nnnnnnn.nnnnnnn}
\startPage{1}

\setcopyright{none}
\bibliographystyle{ACM-Reference-Format}
\citestyle{acmauthoryear}   %% For author/year citations

\begin{document}

\title{Tries that match}

\author{Sebastian Graf}
\affiliation{%
  \institution{Karlsruhe Institute of Technology}
  \city{Karlsruhe}
  \country{Germany}
}
\email{sebastian.graf@kit.edu}

\begin{abstract}
In applications such as compilers and theorem provers, we often want to match
a target term against multiple patterns (representing rewrite rules or axioms)
simultaneously. Efficient matching of this kind is well studied in the theorem prover
community, but much less so in the context of statically typed functional programming.
Doing so is the topic of this talk and yields an interesting new viewpoint
-- one marrying up datatype-generic derivation of generalized tries
\cite{hinze:generalized} with datatype-generic operations for using said tries
as a term index data structure, providing efficient matching lookup of a query
term against indexed patterns.

We start by reviewing Ralph Hinze's work ,,Generalizing generalized
tries'' \cite{hinze:generalized} as an introduction to both polytypic (or
datatype-generic) programming in Haskell and \emph{generalized tries}: Tries
keyed by data types instead of strings of symbols. After studying examples for
some concrete data types and their corresponding lookup function, we examine how
Hinze manages to derive generalized tries for arbitrary first-order polymorphic
datatypes.

After this review, we discuss how to extend the polytypic formulation to cope
with syntax trees, that is, datatypes with variable occurrences and binding
constructs. The challenge is in performing lookup modulo alpha-equivalence by
doing de Bruijn numbering on the fly, as well as in encoding the binding constructs
in the datatype-generic sums-of-products schema language.

Finally we go beyond one-to-one, exact lookups by storing (pattern,value)
associations in our trie data structure, where each pattern may specify and
mention a number of (first-order) unification variables. We use this trie as a
\emph{term index}, where a query with a given ground term against the index
retrieves all (pattern,value) pairs of which the query term is a unification
instance. Our unique achievement is that the query term is not copied during
look up.

As expected, performance measurements indicate that our tries can significantly
outperform ordered maps and hash maps for exact lookups of terms.
We haven't yet put our matching triemap implementation to use in the Glasgow
Haskell Compiler (GHC), but eventually plan to do so, replacing linear searches
in the process.

We conclude by comparing our encoding of a term index to other Haskell
libraries, as well as to \emph{discrimination trees} \cite{mccune:disctree},
the encoding that is closest to our trie data structure in automated reasoning
literature. We'll see that the dataype-generic encoding in Haskell admits
(minor) safety guarantees compared to the untyped string representation of the
term the discrimination tree operates on. The generic encoding also allows for
bespoke data structures depending on the term sort.

\end{abstract}



\maketitle

lkjlkj

\bibliography{refs}

\end{document}
